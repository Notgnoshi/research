% !TEX root = ../notes.tex
\section{Data Sources and Cleaning}\label{sec:data}

{\color{red}
      Find haiku dataset, or for that matter determine if haikus are worth generating.
      Maybe some other form of poetry would work too.

      All of the following \textit{should} be done programmatically and reproduceably.
      \begin{itemize}
            \item Download data
                  \begin{itemize}
                        \item There are still several websites that I have not scraped yet.
                        \item I think this step might require manual labor to get the haikus into a standardized form
                  \end{itemize}
            \item Clean data
                  \begin{itemize}
                        \item Remove non-haikus
                        \item Remove duplicates
                        \item Remove haikus too similar to one another (levenshtein or cosine similarity)
                        \item Restrict the alphabet
                  \end{itemize}
            \item Put it into a standard format
      \end{itemize}

      I also need to figure out how to cite the haikus and their sources.
}

I found a number of websites that publish haikus.
I was able to find enough that my previous concerns about the size of the dataset are no longer relevant.
However, the script I wrote to scrape the websites has a few limitations.

\begin{itemize}
      \item The websites do not consistently (even within the same site) format the haikus in the HTML markup.
            This means that the scraping script has to be very flexible in what it considers a haiku, or that it should only use sources that rigorously format their data.

            If the script is flexible in its haiku classification, there are a lot of false positives, because the markup for a haiku is identical to the markup for potions of the table of contents, for example.

            Limiting the sources to those who are consistent with their formatting cuts the scraped dataset by half.
      \item Again, due to the site inconsistency, there are a lot of false positives, and almost just as many haikus that are missed.
      \item There are a lot of haikus with weird formatting --- resulting in HTML fragments in the haiku, weird punctuation, non-words, and author names and dates in the dataset.

            Also, the markup on some sites does not preserve the concept of distinct lines, and at times includes the author name in the haiku.
      \item Some of the words are not separated by spaces. They are separated by punctuation, but simply replacing punctuation with spaces is not the correct approach.
\end{itemize}

Unfortunately, I currently believe that the above limitations will force me to manually filter and clean the dataset to insert appropriate newlines, clean broken HTML fragments, non-haikus, and broken punctuation.

Perhaps while manually doing the above I will find a programmatic method of cleaning and standardizing the data.

The exploratory data analysis from \autoref{sec:explore} will provide better hints on how to remove outliers, standardize, and clean the dataset.
