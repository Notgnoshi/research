% !TEX root = ../notes.tex
\section{Natural Language Processing}\label{sec:nlp}

\todo{Cite Stanford course.}{\gls{NLP} has different levels.}
First, there is the input layer. This is commonly either text or speech. Text
is then processed, either with OCR or some kind of tokenizer. Speech is
processed via \gls{phonological} analysis. Second, there is a
\gls{morphological} analysis, where the parts of individual words are
analyzed. Third, there is \gls{syntactic} analysis where the structure of
whole sentences or phrases (collections of multiple words) is studied. Fourth,
there is \gls{semantic} interpretation, where the \textit{meaning} of the
sentence(s) are studied. This is much more difficult, because meaning can be
(and often is) divorced from the meanings of the individual words themselves.
That is, the meaning of the whole is often greater than the collected meanings
of the parts.
