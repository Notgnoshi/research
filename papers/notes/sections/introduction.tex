% !TEX root = ../notes.tex
\section{Introduction}\label{sec:introduction}

\subsection{General Research Question}\label{sec:research-question}

This will take some serious refinement, but for now, my research question is

\begin{quote}
    Is it possible to generate believable poetry without falling back to a simple template or rule-based system?
    What techniques in deep learning or Natural Language Processing (NLP) might be employable?
\end{quote}

I will also have to decide if I want to focus on haikus (I think I should at the moment) or some other form of poetry.
Shakespearean sonnets, for example, might be easier because they are longer, and the training dataset is better defined and easier to get a hold of.

\subsection{Techniques to Explore}\label{sec:techniques-to-explore}

I have already attempted to generate haikus using a naive LSTM generative deep learning model.
I did not have much success, but that may have been due to the limited size of my training dataset.

The problem with the LSTM generative model is that it is completely unaware that it is generating \textit{natural language}.
So we are, in effect, not using all of the information available.
That is, haikus follow (loosely) the usual rules of English grammar.
This means that it may be possible to supplement the LSTM model using NLP techniques.

I am currently taking a course in Natural Computing, which excels in areas where the problem is subtle and complex.
Further, it is possible in natural computing to run an algorithm without an explicit fitness function; all that's necessary is a comparative process where one individual is compared to another.
This could be crowdsourced via a website.

One of the most promising techniques that I have learned so far is grammatical evolution via an intermediate solution representation with some set of production rules.
To use this method, I would have to define a grammar for haikus --- that is, find some way to deterministically turn a finite array of discrete symbols into a haiku.

The other problem that I have to solve is that I will need to find a fitness function.
This will have to be done no matter what approach I take.
I believe that it will likely have to be a comparison of two haikus, rather than a strict numerical score between 0 and 1.
