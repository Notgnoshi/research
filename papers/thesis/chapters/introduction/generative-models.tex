% !TeX root = .//.tex
\section{Generative Machine Learning}\label{introduction:generative-models}

As with any field, tasks and models in machine learning can be partitioned with broad strokes by any number of categorization schemes.
One such scheme is the partitioning of models into \textit{discriminative} and \textit{generative} models.

\TODO{Describe each category, and examine their differences, including the differences in the structure of their datasets, their training, and their interpretation.}

\subsection{Generative Adversarial Networks}\label{introduction:generative-models:gans}

\TODO{Read about, then describe how GANs work in detail.
    Especially make note of their requirement for data pulled from a continuous space.
}

Look at
\begin{itemize}
    \item \url{https://arxiv.org/abs/1511.06349}
    \item \url{https://arxiv.org/abs/1511.06038}
    \item \url{https://arxiv.org/abs/1511.06349}
    \item \url{https://arxiv.org/abs/1506.03099}
    \item \url{https://arxiv.org/abs/1705.10929}
    \item \url{https://arxiv.org/abs/1706.03850}
    \item \url{https://akshaybudhkar.com/2018/03/26/generative-adversarial-networks-gans-for-text-using-word2vec/}
\end{itemize}
