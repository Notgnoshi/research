% !TeX root = .//.tex
\section{Text Generation}\label{sec:introduction:text-generation}

\subsection{Recurrent Neural Networks}\label{sec:text-generation:rnns}
\subsection{textgenrnn}\label{sec:text-generation:textgenrnn}
\subsection{Generative Adversarial Networks}\label{sec:text-generation:gans}

\TODO{Read about, then describe how GANs work in detail.
    Especially make note of their requirement for data pulled from a continuous space.
}

Look at
\begin{itemize}
    \item \url{https://arxiv.org/abs/1511.06349}
    \item \url{https://arxiv.org/abs/1511.06038}
    \item \url{https://arxiv.org/abs/1511.06349}
    \item \url{https://arxiv.org/abs/1506.03099}
    \item \url{https://arxiv.org/abs/1705.10929}
    \item \url{https://arxiv.org/abs/1706.03850}
    \item \url{https://akshaybudhkar.com/2018/03/26/generative-adversarial-networks-gans-for-text-using-word2vec/}
\end{itemize}
\TODO{read, summarize, and add to bibliography}
